\documentclass[]{article}
\usepackage{lmodern}
\usepackage{amssymb,amsmath}
\usepackage{ifxetex,ifluatex}
\usepackage{fixltx2e} % provides \textsubscript
\ifnum 0\ifxetex 1\fi\ifluatex 1\fi=0 % if pdftex
  \usepackage[T1]{fontenc}
  \usepackage[utf8]{inputenc}
\else % if luatex or xelatex
  \ifxetex
    \usepackage{mathspec}
  \else
    \usepackage{fontspec}
  \fi
  \defaultfontfeatures{Ligatures=TeX,Scale=MatchLowercase}
\fi
% use upquote if available, for straight quotes in verbatim environments
\IfFileExists{upquote.sty}{\usepackage{upquote}}{}
% use microtype if available
\IfFileExists{microtype.sty}{%
\usepackage{microtype}
\UseMicrotypeSet[protrusion]{basicmath} % disable protrusion for tt fonts
}{}
\usepackage[margin=1in]{geometry}
\usepackage{hyperref}
\hypersetup{unicode=true,
            pdftitle={Estudo de caso: Grupo D 3},
            pdfauthor={Gilmar and Maressa Nunes R. Tavares and Victor},
            pdfborder={0 0 0},
            breaklinks=true}
\urlstyle{same}  % don't use monospace font for urls
\usepackage{color}
\usepackage{fancyvrb}
\newcommand{\VerbBar}{|}
\newcommand{\VERB}{\Verb[commandchars=\\\{\}]}
\DefineVerbatimEnvironment{Highlighting}{Verbatim}{commandchars=\\\{\}}
% Add ',fontsize=\small' for more characters per line
\usepackage{framed}
\definecolor{shadecolor}{RGB}{248,248,248}
\newenvironment{Shaded}{\begin{snugshade}}{\end{snugshade}}
\newcommand{\AlertTok}[1]{\textcolor[rgb]{0.94,0.16,0.16}{#1}}
\newcommand{\AnnotationTok}[1]{\textcolor[rgb]{0.56,0.35,0.01}{\textbf{\textit{#1}}}}
\newcommand{\AttributeTok}[1]{\textcolor[rgb]{0.77,0.63,0.00}{#1}}
\newcommand{\BaseNTok}[1]{\textcolor[rgb]{0.00,0.00,0.81}{#1}}
\newcommand{\BuiltInTok}[1]{#1}
\newcommand{\CharTok}[1]{\textcolor[rgb]{0.31,0.60,0.02}{#1}}
\newcommand{\CommentTok}[1]{\textcolor[rgb]{0.56,0.35,0.01}{\textit{#1}}}
\newcommand{\CommentVarTok}[1]{\textcolor[rgb]{0.56,0.35,0.01}{\textbf{\textit{#1}}}}
\newcommand{\ConstantTok}[1]{\textcolor[rgb]{0.00,0.00,0.00}{#1}}
\newcommand{\ControlFlowTok}[1]{\textcolor[rgb]{0.13,0.29,0.53}{\textbf{#1}}}
\newcommand{\DataTypeTok}[1]{\textcolor[rgb]{0.13,0.29,0.53}{#1}}
\newcommand{\DecValTok}[1]{\textcolor[rgb]{0.00,0.00,0.81}{#1}}
\newcommand{\DocumentationTok}[1]{\textcolor[rgb]{0.56,0.35,0.01}{\textbf{\textit{#1}}}}
\newcommand{\ErrorTok}[1]{\textcolor[rgb]{0.64,0.00,0.00}{\textbf{#1}}}
\newcommand{\ExtensionTok}[1]{#1}
\newcommand{\FloatTok}[1]{\textcolor[rgb]{0.00,0.00,0.81}{#1}}
\newcommand{\FunctionTok}[1]{\textcolor[rgb]{0.00,0.00,0.00}{#1}}
\newcommand{\ImportTok}[1]{#1}
\newcommand{\InformationTok}[1]{\textcolor[rgb]{0.56,0.35,0.01}{\textbf{\textit{#1}}}}
\newcommand{\KeywordTok}[1]{\textcolor[rgb]{0.13,0.29,0.53}{\textbf{#1}}}
\newcommand{\NormalTok}[1]{#1}
\newcommand{\OperatorTok}[1]{\textcolor[rgb]{0.81,0.36,0.00}{\textbf{#1}}}
\newcommand{\OtherTok}[1]{\textcolor[rgb]{0.56,0.35,0.01}{#1}}
\newcommand{\PreprocessorTok}[1]{\textcolor[rgb]{0.56,0.35,0.01}{\textit{#1}}}
\newcommand{\RegionMarkerTok}[1]{#1}
\newcommand{\SpecialCharTok}[1]{\textcolor[rgb]{0.00,0.00,0.00}{#1}}
\newcommand{\SpecialStringTok}[1]{\textcolor[rgb]{0.31,0.60,0.02}{#1}}
\newcommand{\StringTok}[1]{\textcolor[rgb]{0.31,0.60,0.02}{#1}}
\newcommand{\VariableTok}[1]{\textcolor[rgb]{0.00,0.00,0.00}{#1}}
\newcommand{\VerbatimStringTok}[1]{\textcolor[rgb]{0.31,0.60,0.02}{#1}}
\newcommand{\WarningTok}[1]{\textcolor[rgb]{0.56,0.35,0.01}{\textbf{\textit{#1}}}}
\usepackage{graphicx,grffile}
\makeatletter
\def\maxwidth{\ifdim\Gin@nat@width>\linewidth\linewidth\else\Gin@nat@width\fi}
\def\maxheight{\ifdim\Gin@nat@height>\textheight\textheight\else\Gin@nat@height\fi}
\makeatother
% Scale images if necessary, so that they will not overflow the page
% margins by default, and it is still possible to overwrite the defaults
% using explicit options in \includegraphics[width, height, ...]{}
\setkeys{Gin}{width=\maxwidth,height=\maxheight,keepaspectratio}
\IfFileExists{parskip.sty}{%
\usepackage{parskip}
}{% else
\setlength{\parindent}{0pt}
\setlength{\parskip}{6pt plus 2pt minus 1pt}
}
\setlength{\emergencystretch}{3em}  % prevent overfull lines
\providecommand{\tightlist}{%
  \setlength{\itemsep}{0pt}\setlength{\parskip}{0pt}}
\setcounter{secnumdepth}{0}
% Redefines (sub)paragraphs to behave more like sections
\ifx\paragraph\undefined\else
\let\oldparagraph\paragraph
\renewcommand{\paragraph}[1]{\oldparagraph{#1}\mbox{}}
\fi
\ifx\subparagraph\undefined\else
\let\oldsubparagraph\subparagraph
\renewcommand{\subparagraph}[1]{\oldsubparagraph{#1}\mbox{}}
\fi

%%% Use protect on footnotes to avoid problems with footnotes in titles
\let\rmarkdownfootnote\footnote%
\def\footnote{\protect\rmarkdownfootnote}

%%% Change title format to be more compact
\usepackage{titling}

% Create subtitle command for use in maketitle
\providecommand{\subtitle}[1]{
  \posttitle{
    \begin{center}\large#1\end{center}
    }
}

\setlength{\droptitle}{-2em}

  \title{Estudo de caso: Grupo D 3}
    \pretitle{\vspace{\droptitle}\centering\huge}
  \posttitle{\par}
    \author{Gilmar and Maressa Nunes R. Tavares and Victor}
    \preauthor{\centering\large\emph}
  \postauthor{\par}
      \predate{\centering\large\emph}
  \postdate{\par}
    \date{March 00, 2015}


\begin{document}
\maketitle

\begin{verbatim}
## Registered S3 method overwritten by 'GGally':
##   method from   
##   +.gg   ggplot2
\end{verbatim}

\hypertarget{summary}{%
\subsection{Summary}\label{summary}}

O presente trabalho tem como objetivo delinear e executar testes
estatísticos para avaliar uma nova versão de um software, em relação aos
resultados obtidos na versão anterior. Tendo em vista que a última
versão possui uma distribuição do custo computacional com média \$\mu =
50 e variância \$\sigma = 100, dados da população, objetiva-se verificar
se a nova versão apresenta resultados melhores para tais
características. Para tanto, utilizou-se o teste z com nível de
significância \$\alpha = 0,01 e \$\alpha = 0,05, para os testes de média
e variância, respectivamente. Após os testes verificou-se que\ldots{}..

\hypertarget{the-easy-way}{%
\subsubsection{1. The easy way}\label{the-easy-way}}

Use \href{http://rstudio.com/}{RStudio} as your editor, open the
\textbf{.Rmd} file and click the \emph{Knit PDF} button at the top of
the editor.

\hypertarget{the-slightly-less-easy-way}{%
\subsubsection{2. The slightly-less-easy
way}\label{the-slightly-less-easy-way}}

If you're using any other R editor (such as the basic
\href{http://cran.r-project.org}{R} editor), you have to use the
\emph{render()} function from the \textbf{rmarkdown} package:

\begin{Shaded}
\begin{Highlighting}[]
\KeywordTok{install.packages}\NormalTok{(}\StringTok{"devtools"}\NormalTok{)                    }\CommentTok{# you only have to install it once}
\KeywordTok{library}\NormalTok{(devtools)}
\KeywordTok{install_github}\NormalTok{(}\StringTok{"rstudio/rmarkdown"}\NormalTok{)             }\CommentTok{# you only have to install it once}
\KeywordTok{library}\NormalTok{(rmarkdown)}
\KeywordTok{render}\NormalTok{(}\StringTok{"report_template.Rmd"}\NormalTok{,}\StringTok{"pdf_document"}\NormalTok{)    }\CommentTok{# this renders the pdf}
\end{Highlighting}
\end{Shaded}

\hypertarget{experimental-design}{%
\subsection{Experimental design}\label{experimental-design}}

A section detailing the experimental setup. This is the place where you
will define your test hypotheses, e.g.:
\[\begin{cases} H_0: \mu = 10&\\H_1: \mu<10\end{cases}\]

including the reasons behind your choices of the value for \(H_0\) and
the directionality (or not) of \(H_1\).

This is also the place where you should discuss (whenever necessary)
your definitions of minimally relevant effects (\(\delta^*\)), sample
size calculations, choice of power and significance levels, and any
other relevant information about specificities in your data collection
procedures.

\hypertarget{description-of-the-data-collection}{%
\subsubsection{Description of the data
collection}\label{description-of-the-data-collection}}

Whenever needed, you can also include an (optional) subsection
describing the actual data collection, how it was performed, any
adaptations or unexpected events that may have occurred, etc.
Subsections like this can also be used for the sample size calculations
or any other aspect that requires a longer discussion.

\hypertarget{exploratory-data-analysis}{%
\subsection{Exploratory Data Analysis}\label{exploratory-data-analysis}}

The first step is to load and preprocess the data. For instance,

\begin{Shaded}
\begin{Highlighting}[]
\KeywordTok{data}\NormalTok{(mtcars)}
\NormalTok{fc<-}\KeywordTok{c}\NormalTok{(}\DecValTok{2}\NormalTok{,}\DecValTok{8}\OperatorTok{:}\DecValTok{11}\NormalTok{)}
\ControlFlowTok{for}\NormalTok{ (i }\ControlFlowTok{in} \DecValTok{1}\OperatorTok{:}\KeywordTok{length}\NormalTok{(fc))\{mtcars[,fc[i]]<-}\KeywordTok{as.factor}\NormalTok{(mtcars[,fc[i]])\}}
\KeywordTok{levels}\NormalTok{(mtcars}\OperatorTok{$}\NormalTok{am) <-}\StringTok{ }\KeywordTok{c}\NormalTok{(}\StringTok{"Automatic"}\NormalTok{,}\StringTok{"Manual"}\NormalTok{)}
\end{Highlighting}
\end{Shaded}

To get an initial feel for the relationships between the relevant
variables of your experiment it is frequently interesting to perform
some preliminary (exploratory) analysis. This is frequently referred to
as \emph{getting a feel} of your data, and can suggest procedures (such
as outlier investigation or data transformations) to experienced
experimenters.

\begin{Shaded}
\begin{Highlighting}[]
\KeywordTok{library}\NormalTok{(GGally,}\DataTypeTok{quietly =}\NormalTok{ T, }\DataTypeTok{warn.conflicts =}\NormalTok{ F) }\CommentTok{# This is just me getting fancy.}
                                                \CommentTok{# There are much simpler ways ;-)}
\KeywordTok{ggpairs}\NormalTok{(}\DataTypeTok{data=}\NormalTok{mtcars,}\DataTypeTok{columns=}\KeywordTok{c}\NormalTok{(}\DecValTok{1}\NormalTok{,}\DecValTok{9}\NormalTok{),}\DataTypeTok{title=}\StringTok{"MPG by transmission type"}\NormalTok{,}
        \DataTypeTok{upper=}\KeywordTok{list}\NormalTok{(}\DataTypeTok{combo=}\StringTok{"box"}\NormalTok{),}\DataTypeTok{lower=}\KeywordTok{list}\NormalTok{(}\DataTypeTok{combo=}\StringTok{"facethist"}\NormalTok{),}
        \DataTypeTok{diag=}\KeywordTok{list}\NormalTok{(}\DataTypeTok{continuous=}\StringTok{"density"}\NormalTok{,}\DataTypeTok{discrete=}\StringTok{"bar"}\NormalTok{))}
\end{Highlighting}
\end{Shaded}

\begin{verbatim}
## Warning in check_and_set_ggpairs_defaults("diag", diag, continuous =
## "densityDiag", : Changing diag$continuous from 'density' to 'densityDiag'
\end{verbatim}

\begin{verbatim}
## Warning in check_and_set_ggpairs_defaults("diag", diag, continuous =
## "densityDiag", : Changing diag$discrete from 'bar' to 'barDiag'
\end{verbatim}

\begin{figure}
\centering
\includegraphics{CS01report_files/figure-latex/pairs-1.pdf}
\caption{Exploring the effect of car transmission on mpg values}
\end{figure}

Your preliminary analysis should be described together with the plots.
In this example, two facts are immediately clear from the plots: first,
\textbf{mpg} tends to correlate well with many of the other variables,
most intensely with \textbf{drat} (positively) and \textbf{wt}
(negatively). It is also clear that many of the variables are highly
correlated (e.g., \textbf{wt} and \textbf{disp}). Second, it seems like
manual transmission models present larger values of \textbf{mpg} than
the automatic ones. In the next section a linear model will be fit to
the data in order to investigate the significance and magnitude of this
possible effect.

\hypertarget{statistical-analysis}{%
\subsection{Statistical Analysis}\label{statistical-analysis}}

Your statistical analysis should come here. This is the place where you
should fit your statistical model, get the results of your significance
test, your effect size estimates and confidence intervals.

\begin{Shaded}
\begin{Highlighting}[]
\NormalTok{model<-}\KeywordTok{aov}\NormalTok{(mpg}\OperatorTok{~}\NormalTok{am}\OperatorTok{*}\NormalTok{disp,}\DataTypeTok{data=}\NormalTok{mtcars)}
\KeywordTok{summary}\NormalTok{(model)}
\end{Highlighting}
\end{Shaded}

\begin{verbatim}
##             Df Sum Sq Mean Sq F value   Pr(>F)    
## am           1  405.2   405.2  47.948 1.58e-07 ***
## disp         1  420.6   420.6  49.778 1.13e-07 ***
## am:disp      1   63.7    63.7   7.537   0.0104 *  
## Residuals   28  236.6     8.4                     
## ---
## Signif. codes:  0 '***' 0.001 '**' 0.01 '*' 0.05 '.' 0.1 ' ' 1
\end{verbatim}

\hypertarget{checking-model-assumptions}{%
\subsubsection{Checking Model
Assumptions}\label{checking-model-assumptions}}

The assumptions of your test should also be validated, and possible
effects of violations should also be explored.

\begin{Shaded}
\begin{Highlighting}[]
\KeywordTok{par}\NormalTok{(}\DataTypeTok{mfrow=}\KeywordTok{c}\NormalTok{(}\DecValTok{2}\NormalTok{,}\DecValTok{2}\NormalTok{), }\DataTypeTok{mai=}\NormalTok{.}\DecValTok{3}\OperatorTok{*}\KeywordTok{c}\NormalTok{(}\DecValTok{1}\NormalTok{,}\DecValTok{1}\NormalTok{,}\DecValTok{1}\NormalTok{,}\DecValTok{1}\NormalTok{))}
\KeywordTok{plot}\NormalTok{(model,}\DataTypeTok{pch=}\DecValTok{16}\NormalTok{,}\DataTypeTok{lty=}\DecValTok{1}\NormalTok{,}\DataTypeTok{lwd=}\DecValTok{2}\NormalTok{)}
\end{Highlighting}
\end{Shaded}

\begin{figure}
\centering
\includegraphics{CS01report_files/figure-latex/resplots-1.pdf}
\caption{Residual plots for the anova model}
\end{figure}

\hypertarget{conclusions-and-recommendations}{%
\subsubsection{Conclusions and
Recommendations}\label{conclusions-and-recommendations}}

The discussion of your results, and the scientific/technical meaning of
the effects detected, should be placed here. Always be sure to tie your
results back to the original question of interest!


\end{document}
