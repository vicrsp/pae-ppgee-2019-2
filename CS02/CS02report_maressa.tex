\documentclass[]{article}
\usepackage{lmodern}
\usepackage{amssymb,amsmath}
\usepackage{ifxetex,ifluatex}
\usepackage{fixltx2e} % provides \textsubscript
\ifnum 0\ifxetex 1\fi\ifluatex 1\fi=0 % if pdftex
  \usepackage[T1]{fontenc}
  \usepackage[utf8]{inputenc}
\else % if luatex or xelatex
  \ifxetex
    \usepackage{mathspec}
  \else
    \usepackage{fontspec}
  \fi
  \defaultfontfeatures{Ligatures=TeX,Scale=MatchLowercase}
\fi
% use upquote if available, for straight quotes in verbatim environments
\IfFileExists{upquote.sty}{\usepackage{upquote}}{}
% use microtype if available
\IfFileExists{microtype.sty}{%
\usepackage{microtype}
\UseMicrotypeSet[protrusion]{basicmath} % disable protrusion for tt fonts
}{}
\usepackage[margin=1in]{geometry}
\usepackage{hyperref}
\hypersetup{unicode=true,
            pdftitle={Estudo de caso: Grupo D 3},
            pdfauthor={Gilmar and Maressa Nunes R. Tavares and Victor},
            pdfborder={0 0 0},
            breaklinks=true}
\urlstyle{same}  % don't use monospace font for urls
\usepackage{color}
\usepackage{fancyvrb}
\newcommand{\VerbBar}{|}
\newcommand{\VERB}{\Verb[commandchars=\\\{\}]}
\DefineVerbatimEnvironment{Highlighting}{Verbatim}{commandchars=\\\{\}}
% Add ',fontsize=\small' for more characters per line
\usepackage{framed}
\definecolor{shadecolor}{RGB}{248,248,248}
\newenvironment{Shaded}{\begin{snugshade}}{\end{snugshade}}
\newcommand{\AlertTok}[1]{\textcolor[rgb]{0.94,0.16,0.16}{#1}}
\newcommand{\AnnotationTok}[1]{\textcolor[rgb]{0.56,0.35,0.01}{\textbf{\textit{#1}}}}
\newcommand{\AttributeTok}[1]{\textcolor[rgb]{0.77,0.63,0.00}{#1}}
\newcommand{\BaseNTok}[1]{\textcolor[rgb]{0.00,0.00,0.81}{#1}}
\newcommand{\BuiltInTok}[1]{#1}
\newcommand{\CharTok}[1]{\textcolor[rgb]{0.31,0.60,0.02}{#1}}
\newcommand{\CommentTok}[1]{\textcolor[rgb]{0.56,0.35,0.01}{\textit{#1}}}
\newcommand{\CommentVarTok}[1]{\textcolor[rgb]{0.56,0.35,0.01}{\textbf{\textit{#1}}}}
\newcommand{\ConstantTok}[1]{\textcolor[rgb]{0.00,0.00,0.00}{#1}}
\newcommand{\ControlFlowTok}[1]{\textcolor[rgb]{0.13,0.29,0.53}{\textbf{#1}}}
\newcommand{\DataTypeTok}[1]{\textcolor[rgb]{0.13,0.29,0.53}{#1}}
\newcommand{\DecValTok}[1]{\textcolor[rgb]{0.00,0.00,0.81}{#1}}
\newcommand{\DocumentationTok}[1]{\textcolor[rgb]{0.56,0.35,0.01}{\textbf{\textit{#1}}}}
\newcommand{\ErrorTok}[1]{\textcolor[rgb]{0.64,0.00,0.00}{\textbf{#1}}}
\newcommand{\ExtensionTok}[1]{#1}
\newcommand{\FloatTok}[1]{\textcolor[rgb]{0.00,0.00,0.81}{#1}}
\newcommand{\FunctionTok}[1]{\textcolor[rgb]{0.00,0.00,0.00}{#1}}
\newcommand{\ImportTok}[1]{#1}
\newcommand{\InformationTok}[1]{\textcolor[rgb]{0.56,0.35,0.01}{\textbf{\textit{#1}}}}
\newcommand{\KeywordTok}[1]{\textcolor[rgb]{0.13,0.29,0.53}{\textbf{#1}}}
\newcommand{\NormalTok}[1]{#1}
\newcommand{\OperatorTok}[1]{\textcolor[rgb]{0.81,0.36,0.00}{\textbf{#1}}}
\newcommand{\OtherTok}[1]{\textcolor[rgb]{0.56,0.35,0.01}{#1}}
\newcommand{\PreprocessorTok}[1]{\textcolor[rgb]{0.56,0.35,0.01}{\textit{#1}}}
\newcommand{\RegionMarkerTok}[1]{#1}
\newcommand{\SpecialCharTok}[1]{\textcolor[rgb]{0.00,0.00,0.00}{#1}}
\newcommand{\SpecialStringTok}[1]{\textcolor[rgb]{0.31,0.60,0.02}{#1}}
\newcommand{\StringTok}[1]{\textcolor[rgb]{0.31,0.60,0.02}{#1}}
\newcommand{\VariableTok}[1]{\textcolor[rgb]{0.00,0.00,0.00}{#1}}
\newcommand{\VerbatimStringTok}[1]{\textcolor[rgb]{0.31,0.60,0.02}{#1}}
\newcommand{\WarningTok}[1]{\textcolor[rgb]{0.56,0.35,0.01}{\textbf{\textit{#1}}}}
\usepackage{graphicx,grffile}
\makeatletter
\def\maxwidth{\ifdim\Gin@nat@width>\linewidth\linewidth\else\Gin@nat@width\fi}
\def\maxheight{\ifdim\Gin@nat@height>\textheight\textheight\else\Gin@nat@height\fi}
\makeatother
% Scale images if necessary, so that they will not overflow the page
% margins by default, and it is still possible to overwrite the defaults
% using explicit options in \includegraphics[width, height, ...]{}
\setkeys{Gin}{width=\maxwidth,height=\maxheight,keepaspectratio}
\IfFileExists{parskip.sty}{%
\usepackage{parskip}
}{% else
\setlength{\parindent}{0pt}
\setlength{\parskip}{6pt plus 2pt minus 1pt}
}
\setlength{\emergencystretch}{3em}  % prevent overfull lines
\providecommand{\tightlist}{%
  \setlength{\itemsep}{0pt}\setlength{\parskip}{0pt}}
\setcounter{secnumdepth}{5}
% Redefines (sub)paragraphs to behave more like sections
\ifx\paragraph\undefined\else
\let\oldparagraph\paragraph
\renewcommand{\paragraph}[1]{\oldparagraph{#1}\mbox{}}
\fi
\ifx\subparagraph\undefined\else
\let\oldsubparagraph\subparagraph
\renewcommand{\subparagraph}[1]{\oldsubparagraph{#1}\mbox{}}
\fi

%%% Use protect on footnotes to avoid problems with footnotes in titles
\let\rmarkdownfootnote\footnote%
\def\footnote{\protect\rmarkdownfootnote}

%%% Change title format to be more compact
\usepackage{titling}

% Create subtitle command for use in maketitle
\providecommand{\subtitle}[1]{
  \posttitle{
    \begin{center}\large#1\end{center}
    }
}

\setlength{\droptitle}{-2em}

  \title{Estudo de caso: Grupo D 3}
    \pretitle{\vspace{\droptitle}\centering\huge}
  \posttitle{\par}
    \author{Gilmar and Maressa Nunes R. Tavares and Victor}
    \preauthor{\centering\large\emph}
  \postauthor{\par}
      \predate{\centering\large\emph}
  \postdate{\par}
    \date{3 de Setembro, 2019}


\begin{document}
\maketitle

\hypertarget{summary}{%
\section{Summary}\label{summary}}

\hypertarget{planejamento-do-experimento}{%
\section{Planejamento do
experimento}\label{planejamento-do-experimento}}

\hypertarget{objetivo-do-experimento}{%
\subsection{Objetivo do experimento}\label{objetivo-do-experimento}}

\[\begin{cases} H_0: \mu = 50&\\H_1: \mu<50\end{cases}\]

\[\begin{cases} H_0: \sigma^{2} = 100&\\H_1: \sigma^{2} < 100\end{cases}\]

\hypertarget{descricao-da-coleta-de-dados}{%
\subsubsection{Descrição da coleta de
dados}\label{descricao-da-coleta-de-dados}}

\hypertarget{descricao-da-coleta-de-dados-1}{%
\subsubsection{Descrição da coleta de
dados}\label{descricao-da-coleta-de-dados-1}}

\begin{Shaded}
\begin{Highlighting}[]
\NormalTok{imc <-}\StringTok{ }\ControlFlowTok{function}\NormalTok{(height, weight)\{}
  \KeywordTok{return}\NormalTok{(weight}\OperatorTok{/}\NormalTok{(height}\OperatorTok{^}\DecValTok{2}\NormalTok{))}
\NormalTok{\}}

\NormalTok{data.}\FloatTok{2016.2}\NormalTok{ <-}\StringTok{ }\KeywordTok{read.csv}\NormalTok{(}\StringTok{'imc_20162.csv'}\NormalTok{, }\DataTypeTok{header =}\NormalTok{ T, }\DataTypeTok{sep =} \StringTok{","}\NormalTok{)}
\NormalTok{data.}\FloatTok{2017.2}\NormalTok{ <-}\StringTok{ }\KeywordTok{read.csv}\NormalTok{(}\StringTok{'CS01_20172.csv'}\NormalTok{, }\DataTypeTok{header =}\NormalTok{ T, }\DataTypeTok{sep =} \StringTok{";"}\NormalTok{)}
\NormalTok{data.}\FloatTok{2017.2}\NormalTok{ <-}\StringTok{ }\NormalTok{data.}\FloatTok{2017.2} \OperatorTok\StringTok{ }\KeywordTok{rename}\NormalTok{(}\DataTypeTok{Gender =}\NormalTok{ Sex)}

\CommentTok{# calculate the IMC}
\NormalTok{data.}\FloatTok{2016.2}\OperatorTok{$}\NormalTok{imc <-}\StringTok{ }\KeywordTok{imc}\NormalTok{(data.}\FloatTok{2016.2}\OperatorTok{$}\NormalTok{Height.m, data.}\FloatTok{2016.2}\OperatorTok{$}\NormalTok{Weight.kg)}
\NormalTok{data.}\FloatTok{2017.2}\OperatorTok{$}\NormalTok{imc <-}\StringTok{ }\KeywordTok{imc}\NormalTok{(data.}\FloatTok{2017.2}\OperatorTok{$}\NormalTok{height.m, data.}\FloatTok{2017.2}\OperatorTok{$}\NormalTok{Weight.kg)}

\CommentTok{# remove undergraduate students for 2016.2 data}
\NormalTok{data.}\FloatTok{2016.2}\NormalTok{ <-}\StringTok{ }\NormalTok{data.}\FloatTok{2016.2} \OperatorTok\StringTok{ }\KeywordTok{filter}\NormalTok{(Course }\OperatorTok{==}\StringTok{ 'PPGEE'}\NormalTok{)}

\CommentTok{# remove unecessary columns and rename}
\NormalTok{data.}\FloatTok{2016.2}\NormalTok{ <-}\StringTok{ }\NormalTok{data.}\FloatTok{2016.2} \OperatorTok\StringTok{ }\KeywordTok{select}\NormalTok{(Gender, imc)}
\NormalTok{data.}\FloatTok{2017.2}\NormalTok{ <-}\StringTok{ }\NormalTok{data.}\FloatTok{2017.2} \OperatorTok\StringTok{ }\KeywordTok{select}\NormalTok{(Gender, imc)}

\CommentTok{# split the 2016.2 data by gender and remove undergraduate students}
\NormalTok{data.}\DecValTok{2016}\NormalTok{.}\FloatTok{2.}\NormalTok{Females <-}\StringTok{ }\NormalTok{data.}\FloatTok{2016.2} \OperatorTok\StringTok{ }\KeywordTok{filter}\NormalTok{(Gender }\OperatorTok{==}\StringTok{ 'F'}\NormalTok{) }\OperatorTok\StringTok{ }\KeywordTok{select}\NormalTok{(imc)}
\NormalTok{data.}\DecValTok{2016}\NormalTok{.}\FloatTok{2.}\NormalTok{Males <-}\StringTok{ }\NormalTok{data.}\FloatTok{2016.2} \OperatorTok\StringTok{ }\KeywordTok{filter}\NormalTok{(Gender }\OperatorTok{==}\StringTok{ 'M'}\NormalTok{) }\OperatorTok\StringTok{ }\KeywordTok{select}\NormalTok{(imc)}

\CommentTok{# split the 2017.2 data by gender}
\NormalTok{data.}\DecValTok{2017}\NormalTok{.}\FloatTok{2.}\NormalTok{Females <-}\StringTok{ }\NormalTok{data.}\FloatTok{2017.2} \OperatorTok\StringTok{ }\KeywordTok{filter}\NormalTok{(Gender }\OperatorTok{==}\StringTok{ 'F'}\NormalTok{) }\OperatorTok\StringTok{ }\KeywordTok{select}\NormalTok{(imc)}
\NormalTok{data.}\DecValTok{2017}\NormalTok{.}\FloatTok{2.}\NormalTok{Males <-}\StringTok{ }\NormalTok{data.}\FloatTok{2017.2} \OperatorTok\StringTok{ }\KeywordTok{filter}\NormalTok{(Gender }\OperatorTok{==}\StringTok{ 'M'}\NormalTok{) }\OperatorTok\StringTok{ }\KeywordTok{select}\NormalTok{(imc)}
\end{Highlighting}
\end{Shaded}

\hypertarget{analise-exploratoria-dos-dados}{%
\section{Análise Exploratória dos
Dados}\label{analise-exploratoria-dos-dados}}

\hypertarget{resultados}{%
\section{Resultados}\label{resultados}}

\hypertarget{validacao-das-premissas}{%
\subsection{Validação das premissas}\label{validacao-das-premissas}}

Para realizar as inferências estatísticas sobre o IMC das duas
populações é necessário validar as premissas antes de executar o teste.
Neste caso, como as variâncias das duas populações é desconhecida,
utiliza-se a distribuição t para o teste de hipóteses e para os
intervalos de confiança {[}1{]}. A seguir são apresentados os testes
realizados para validar as premissas exigidas pelo teste t. Para
facilitar as análises optou-se por separar o grupo na população feminina
e masculina.

\hypertarget{subpopulacao-feminina}{%
\subsubsection{Subpopulação Feminina}\label{subpopulacao-feminina}}

1 - Normalidade:

Para avaliar a normalidade dos dados das duas subpopulações, utilizou-se
o gráfico quantil-quantil e o teste de Shapiro Wilk com
\(\alpha = 0.05\), como apresnetado a seguir.

\begin{verbatim}
## [1] 6 1
\end{verbatim}

\begin{figure}
\centering
\includegraphics{CS02report_maressa_files/figure-latex/plot_norm_fem-1.pdf}
\caption{Gráfico quantil-quantil das populações - Feminino}
\end{figure}

\begin{verbatim}
## [1] 2 1
\end{verbatim}

\begin{Shaded}
\begin{Highlighting}[]
\KeywordTok{shapiro.test}\NormalTok{(data.}\DecValTok{2016}\NormalTok{.}\FloatTok{2.}\NormalTok{Females}\OperatorTok{$}\NormalTok{imc)}
\end{Highlighting}
\end{Shaded}

\begin{verbatim}
## 
##  Shapiro-Wilk normality test
## 
## data:  data.2016.2.Females$imc
## W = 0.91974, p-value = 0.4674
\end{verbatim}

\begin{Shaded}
\begin{Highlighting}[]
\KeywordTok{shapiro.test}\NormalTok{(data.}\DecValTok{2017}\NormalTok{.}\FloatTok{2.}\NormalTok{Females}\OperatorTok{$}\NormalTok{imc)}
\end{Highlighting}
\end{Shaded}

\begin{verbatim}
## 
##  Shapiro-Wilk normality test
## 
## data:  data.2017.2.Females$imc
## W = 0.7475, p-value = 0.03659
\end{verbatim}

Pela análise do teste de Shapiro-Wilk e do gráfico concluímos que não há
evidências para rejeitar a normalidade dos dados de 2016.2, pois o
p-valor do teste de Shapiro Wilk foi maior que 0.05. Por outro lado, os
dados de 2017.2 apresentam um ponto fora dos limites, e por isso, há
evidências para rejeitar a normalidade dos dados.

2- Igualdade de Variâncias:

A segunda premissa a ser avaliada é a igualdade de variâncias das duas
populações, homocedasticidade. Para tanto, utilizou-se o teste F com a
função var.test e \(\alpha = 0.05\), que tem como hipótese nula a
igualdade da variância das duas amostras.

\begin{Shaded}
\begin{Highlighting}[]
\KeywordTok{var.test}\NormalTok{(data.}\DecValTok{2016}\NormalTok{.}\FloatTok{2.}\NormalTok{Females}\OperatorTok{$}\NormalTok{imc,data.}\DecValTok{2017}\NormalTok{.}\FloatTok{2.}\NormalTok{Females}\OperatorTok{$}\NormalTok{imc)}
\end{Highlighting}
\end{Shaded}

\begin{verbatim}
## 
##  F test to compare two variances
## 
## data:  data.2016.2.Females$imc and data.2017.2.Females$imc
## F = 2.2688, num df = 6, denom df = 3, p-value = 0.5353
## alternative hypothesis: true ratio of variances is not equal to 1
## 95 percent confidence interval:
##   0.1539783 14.9715344
## sample estimates:
## ratio of variances 
##           2.268827
\end{verbatim}

Pelo teste F conclui-se que não há evidências para rejeitar a hipótese
nula de igualdade das variâncias das duas populações, portanto, as
amostras são consideradas homocedásticas.

3 - Independência

Considerando que as populações referem-se a dois grupos distintos da
pós-graduação, sabe-se a priori, que as amostras são independentes. Para
concluir com precisão em relação à independência realizou-se o teste
Qui-quadrado com \(\alpha = 0.05\).

\begin{Shaded}
\begin{Highlighting}[]
\NormalTok{data.chi.Female <-}\StringTok{ }\KeywordTok{c}\NormalTok{(data.}\DecValTok{2016}\NormalTok{.}\FloatTok{2.}\NormalTok{Females}\OperatorTok{$}\NormalTok{imc, data.}\DecValTok{2017}\NormalTok{.}\FloatTok{2.}\NormalTok{Females}\OperatorTok{$}\NormalTok{imc)}
\KeywordTok{chisq.test}\NormalTok{(data.chi.Female)}
\end{Highlighting}
\end{Shaded}

\begin{verbatim}
## 
##  Chi-squared test for given probabilities
## 
## data:  data.chi.Female
## X-squared = 3.0047, df = 10, p-value = 0.9813
\end{verbatim}

Como era esperado, o teste Qui-quadrado reafirma a independência entre
as duas amostras, como o pvalor = 0.9565.

\hypertarget{subpopulacao-masculina}{%
\subsubsection{Subpopulação Masculina}\label{subpopulacao-masculina}}

Assim como na validação das premissas da subpopulação feminina, foram
realizados os testes para a subpopulação masculina, como descrito a
seguir.

1 - Normalidade:

Para validar a normalidade dos dados utilizou-se o gráfico
quantil-quantil e o teste de Shapiro Wilk com \(\alpha = 0.05\).

\begin{verbatim}
## [1]  4 12
\end{verbatim}

\begin{figure}
\centering
\includegraphics{CS02report_maressa_files/figure-latex/plot_norm_male-1.pdf}
\caption{Gráfico quantil-quantil das populações - Feminino}
\end{figure}

\begin{verbatim}
## [1] 12  4
\end{verbatim}

\begin{Shaded}
\begin{Highlighting}[]
\KeywordTok{shapiro.test}\NormalTok{(data.}\DecValTok{2016}\NormalTok{.}\FloatTok{2.}\NormalTok{Males}\OperatorTok{$}\NormalTok{imc)}
\end{Highlighting}
\end{Shaded}

\begin{verbatim}
## 
##  Shapiro-Wilk normality test
## 
## data:  data.2016.2.Males$imc
## W = 0.92833, p-value = 0.1275
\end{verbatim}

\begin{Shaded}
\begin{Highlighting}[]
\KeywordTok{shapiro.test}\NormalTok{(data.}\DecValTok{2017}\NormalTok{.}\FloatTok{2.}\NormalTok{Males}\OperatorTok{$}\NormalTok{imc)}
\end{Highlighting}
\end{Shaded}

\begin{verbatim}
## 
##  Shapiro-Wilk normality test
## 
## data:  data.2017.2.Males$imc
## W = 0.96494, p-value = 0.6206
\end{verbatim}

Pela análise do teste de Shapiro-Wilk e dos gráficos concluímos que não
há evidências para rejeitar a hipótese nula para as duas amostras, pois
em ambas o p-valor do teste foi superior a 0.05. Portanto, para a
subpopulação masculina dos dois semestres os dados estão normalmente
distribuídos.

2- Igualdade de Variâncias:

Para validar a igualdade de variâncias das duas amostras utilizou-se o
teste F com \(\alpha = 0.05\). A hipótese nula considera a igualdade da
variância das duas amostras.

\begin{Shaded}
\begin{Highlighting}[]
\KeywordTok{var.test}\NormalTok{(data.}\DecValTok{2016}\NormalTok{.}\FloatTok{2.}\NormalTok{Males}\OperatorTok{$}\NormalTok{imc,data.}\DecValTok{2017}\NormalTok{.}\FloatTok{2.}\NormalTok{Males}\OperatorTok{$}\NormalTok{imc)}
\end{Highlighting}
\end{Shaded}

\begin{verbatim}
## 
##  F test to compare two variances
## 
## data:  data.2016.2.Males$imc and data.2017.2.Males$imc
## F = 1.5839, num df = 20, denom df = 20, p-value = 0.3119
## alternative hypothesis: true ratio of variances is not equal to 1
## 95 percent confidence interval:
##  0.6426853 3.9034665
## sample estimates:
## ratio of variances 
##           1.583888
\end{verbatim}

Pelo teste F conclui-se que as populações são homocedásticas, pois o
p-valor foi superior a 0.05, logo, não há evidências para rejeitar a
hipótese nula.

3 - Independência

Assim como na subpopulação feminina, sabe-se a priori, que as amostras
são independentes. Para concluir com precisão em relação à independência
realizou-se o teste Qui-quadrado com \(\alpha = 0.05\).

\begin{Shaded}
\begin{Highlighting}[]
\NormalTok{data.chi.male <-}\StringTok{ }\KeywordTok{c}\NormalTok{(data.}\DecValTok{2016}\NormalTok{.}\FloatTok{2.}\NormalTok{Males}\OperatorTok{$}\NormalTok{imc,data.}\DecValTok{2017}\NormalTok{.}\FloatTok{2.}\NormalTok{Males}\OperatorTok{$}\NormalTok{imc)}
\KeywordTok{chisq.test}\NormalTok{(data.chi.male)}
\end{Highlighting}
\end{Shaded}

\begin{verbatim}
## 
##  Chi-squared test for given probabilities
## 
## data:  data.chi.male
## X-squared = 24.96, df = 41, p-value = 0.9772
\end{verbatim}

Como era esperado, o teste Qui-quadrado reafirma a independência entre
as duas amostras, como o pvalor = 0.9813.

\hypertarget{discussao-e-conclusoes}{%
\section{Discussão e Conclusões}\label{discussao-e-conclusoes}}

\hypertarget{divisao-das-atividades}{%
\section{Divisão das Atividades}\label{divisao-das-atividades}}

Victor - Reporter Maressa - Coordenadora Gilmar - Verificador e Monitor

\hypertarget{referencias}{%
\section*{Referências}\label{referencias}}
\addcontentsline{toc}{section}{Referências}

\hypertarget{refs}{}
\leavevmode\hypertarget{ref-montgomery2007applied}{}%
{[}1{]} D. C. Montgomery and G. C. Runger, \emph{Applied statistics and
probability for engineers, (with cd)}. John Wiley \& Sons, 2007.


\end{document}
